\documentclass[12pt]{article}

% ---------------------------
% Packages
% ---------------------------
\usepackage[utf8]{inputenc}
\usepackage{amsmath, amssymb, amsthm} % For math symbols
\usepackage{bm} % For bold math
\usepackage{graphicx} % For images
\usepackage{enumitem} % For custom lists
\usepackage{hyperref} % For links
\usepackage{geometry} % Better margins
\geometry{margin=1in}
\usepackage{float}

\usepackage{tikz}
\usetikzlibrary{bayesnet,positioning,calc,fit}
\tikzset{
  fixed/.style={circle, fill=black, inner sep=1.4pt, minimum size=2pt},
}
\usepackage{booktabs}
\usepackage{algorithm}
\usepackage{algorithmic}

% ---------------------------
% Title
% ---------------------------
\title{STCS 6701: Probabilistic Machine Learning \\ Project}
%\author{Rogelio Lopez Camara\\ rl3499}
\date{\today}

\begin{document}

\maketitle
% ============================================================
% 1. GAUSSIAN MATRIX FACTORIZATION (NO BIASES)
% ============================================================

\section{Gaussian Matrix Factorization with Mean-field Variational Inference}

\subsection{Model Definition}

We consider a standard Gaussian matrix factorization model.
Let $U$ be the number of users, $M$ the number of items, and $K$ the latent dimension.
For each centered observation $x_{ij}$ we assume
\[
x_{ij} \mid \theta_i, \beta_j \sim \mathcal{N}(\theta_i^\top \beta_j, \sigma^2),
\]
where user and item latent vectors follow independent isotropic Gaussian priors:
\[
\theta_i \sim \mathcal{N}(0,\eta_\theta^2 I_K),
\qquad
\beta_j \sim \mathcal{N}(0,\eta_\beta^2 I_K).
\]

We denote by
\[
\Omega_i = \{ j : (i,j)\ \text{is observed} \}, 
\qquad
\Omega^j = \{ i : (i,j)\ \text{is observed} \}.
\]

\subsection{Complete Conditionals}

Since the model is conjugate, all conditional distributions remain Gaussian.

\subsubsection{Conditional for the user factor $\theta_i$}

The relevant terms in the joint distribution are
\[
p(\theta_i)\prod_{j \in \Omega_i} p(x_{ij} \mid \theta_i,\beta_j),
\]
leading to a Gaussian conditional with precision
\[
\Lambda_{\theta_i} = 
\eta_\theta^{-2} I_K + \sigma^{-2} \sum_{j\in\Omega_i} \beta_j \beta_j^\top,
\]
and mean
\[
\mu_{\theta_i}
=
\Lambda_{\theta_i}^{-1}
\left(
\sigma^{-2} \sum_{j\in\Omega_i} x_{ij}\,\beta_j
\right).
\]

\subsubsection{Conditional for the item factor $\beta_j$}

By symmetry:
\[
\Lambda_{\beta_j}
=
\eta_\beta^{-2} I_K + \sigma^{-2} \sum_{i\in\Omega^j} \theta_i \theta_i^\top,
\]
\[
\mu_{\beta_j}
=
\Lambda_{\beta_j}^{-1}
\left(
\sigma^{-2} \sum_{i\in\Omega^j} x_{ij}\,\theta_i
\right).
\]

\subsection{Variational Family and CAVI Updates}

We assume a fully factorized Gaussian approximation:
\[
q(\theta,\beta) = 
\prod_{i=1}^U \mathcal{N}(\theta_i \mid m_{\theta_i},V_{\theta_i}) 
\prod_{j=1}^M \mathcal{N}(\beta_j \mid m_{\beta_j},V_{\beta_j}).
\]

The expectations required for CAVI updates are
\[
\mathbb{E}_q[\beta_j] = m_{\beta_j},
\qquad
\mathbb{E}_q[\beta_j \beta_j^\top]
= V_{\beta_j} + m_{\beta_j} m_{\beta_j}^\top,
\]
and similarly for $\theta_i$.

Define summary matrices:
\[
S_i = \sum_{j\in\Omega_i}
\left(
V_{\beta_j} + m_{\beta_j} m_{\beta_j}^\top
\right),
\qquad
T_j = \sum_{i\in\Omega^j}
\left(
V_{\theta_i} + m_{\theta_i} m_{\theta_i}^\top
\right).
\]

\subsubsection{CAVI updates for user factors}

\[
V_{\theta_i}
=
\left(
\eta_\theta^{-2} I_K + \sigma^{-2} S_i
\right)^{-1},
\qquad
m_{\theta_i}
=
V_{\theta_i}
\left(
\sigma^{-2}\sum_{j\in\Omega_i} x_{ij} m_{\beta_j}
\right).
\]

\subsubsection{CAVI updates for item factors}

\[
V_{\beta_j}
=
\left(
\eta_\beta^{-2} I_K + \sigma^{-2} T_j
\right)^{-1},
\qquad
m_{\beta_j}
=
V_{\beta_j}
\left(
\sigma^{-2}\sum_{i\in\Omega^j} x_{ij} m_{\theta_i}
\right).
\]

\subsection{Pseudocode}

\begin{algorithm}[H]
\caption{CAVI for Gaussian Matrix Factorization}
\begin{algorithmic}[1]
\STATE Initialize all variational means and covariances.
\FOR{iteration $t=1,\dots,T_{\max}$}
    \FOR{each user $i$}
        \STATE Compute $S_i$.
        \STATE Update $V_{\theta_i}$ and $m_{\theta_i}$.
    \ENDFOR
    \FOR{each item $j$}
        \STATE Compute $T_j$.
        \STATE Update $V_{\beta_j}$ and $m_{\beta_j}$.
    \ENDFOR
    \STATE Optionally compute validation RMSE.
\ENDFOR
\end{algorithmic}
\end{algorithm}



% ============================================================
% 2. GAUSSIAN MF WITH USER AND ITEM BIASES
% ============================================================

\section{Gaussian Matrix Factorization with User and Item Biases}

\subsection{Model Definition}

We now extend the previous model by introducing user and item biases.
For each centered rating $x_{ij}$ we assume
\[
x_{ij} \mid \theta_i, \beta_j, b_i, b_j 
\sim \mathcal{N}( b_i + b_j + \theta_i^\top \beta_j , \sigma^2 ).
\]

Latent factors keep the same priors as before:
\[
\theta_i \sim \mathcal{N}(0,\eta_\theta^2 I_K),\qquad
\beta_j \sim \mathcal{N}(0,\eta_\beta^2 I_K),
\]
and biases follow
\[
b_i \sim \mathcal{N}(0,\eta_b^2),
\qquad
b_j \sim \mathcal{N}(0,\eta_b^2).
\]

We use the same observation index sets $\Omega_i$ and $\Omega^j$ as in the previous section.

\subsection{Complete Conditionals}

\subsubsection{Conditional for the user factor $\theta_i$}

The Gaussian conditional is
\[
\Lambda_{\theta_i}
=
\eta_\theta^{-2} I_K
+
\sigma^{-2}\sum_{j\in\Omega_i}\beta_j \beta_j^\top,
\]
\[
\mu_{\theta_i}
=
\Lambda_{\theta_i}^{-1}
\left(
\sigma^{-2}\sum_{j\in\Omega_i}(x_{ij}-b_i-b_j)\beta_j
\right).
\]

\subsubsection{Conditional for the item factor $\beta_j$}

\[
\Lambda_{\beta_j}
=
\eta_\beta^{-2}I_K
+
\sigma^{-2}\sum_{i\in\Omega^j}\theta_i\theta_i^\top,
\]
\[
\mu_{\beta_j}
=
\Lambda_{\beta_j}^{-1}
\left(
\sigma^{-2}\sum_{i\in\Omega^j}(x_{ij}-b_i-b_j)\theta_i
\right).
\]

\subsubsection{Conditional for the user bias $b_i$}

\[
\sigma^{-2}_{b_i}
=
\eta_b^{-2} + \frac{|\Omega_i|}{\sigma^2},
\qquad
\mu_{b_i}
=
\sigma^2_{b_i}
\left(
\frac{1}{\sigma^2}
\sum_{j\in\Omega_i}(x_{ij}-b_j-\theta_i^\top\beta_j)
\right).
\]

\subsubsection{Conditional for the item bias $b_j$}

\[
\sigma^{-2}_{b_j}
=
\eta_b^{-2} + \frac{|\Omega^j|}{\sigma^2},
\qquad
\mu_{b_j}
=
\sigma^2_{b_j}
\left(
\frac{1}{\sigma^2}
\sum_{i\in\Omega^j}(x_{ij}-b_i-\theta_i^\top\beta_j)
\right).
\]

\subsection{Variational Family and CAVI Updates}

We adopt the factorization
\[
q = 
\prod_{i=1}^U q(\theta_i) q(b_i)
\prod_{j=1}^M q(\beta_j) q(b_j).
\]

Expectations for latent vectors are identical to the previous model.  
The bias expectations are simply their variational means:
\[
\mathbb{E}_q[b_i] = m_{b_i}, \qquad
\mathbb{E}_q[b_j] = m_{b_j}.
\]

Define the same summary matrices $S_i$ and $T_j$ as in the previous model.

\subsubsection{CAVI updates for user factors}

\[
V_{\theta_i}
=
(\eta_\theta^{-2}I_K + \sigma^{-2} S_i)^{-1},
\qquad
m_{\theta_i}
=
V_{\theta_i}
\left(
\sigma^{-2}\sum_{j\in\Omega_i}(x_{ij}-m_{b_i}-m_{b_j})m_{\beta_j}
\right).
\]

\subsubsection{CAVI updates for item factors}

\[
V_{\beta_j}
=
(\eta_\beta^{-2}I_K + \sigma^{-2}T_j)^{-1},
\qquad
m_{\beta_j}
=
V_{\beta_j}
\left(
\sigma^{-2}\sum_{i\in\Omega^j}(x_{ij}-m_{b_i}-m_{b_j})m_{\theta_i}
\right).
\]

\subsubsection{CAVI updates for user biases}

\[
\sigma^{-2}_{b_i}
=
\eta_b^{-2} + \frac{|\Omega_i|}{\sigma^2},
\qquad
m_{b_i}
=
\sigma^2_{b_i}
\left(
\frac{1}{\sigma^2}
\sum_{j\in\Omega_i}(x_{ij}-m_{b_j}-m_{\theta_i}^\top m_{\beta_j})
\right).
\]

\subsubsection{CAVI updates for item biases}

\[
\sigma^{-2}_{b_j}
=
\eta_b^{-2} + \frac{|\Omega^j|}{\sigma^2},
\qquad
m_{b_j}
=
\sigma^2_{b_j}
\left(
\frac{1}{\sigma^2}
\sum_{i\in\Omega^j}(x_{ij}-m_{b_i}-m_{\theta_i}^\top m_{\beta_j})
\right).
\]

\subsection{Pseudocode}

\begin{algorithm}[H]
\caption{CAVI for Gaussian MF with User and Item Biases}
\begin{algorithmic}[1]
\STATE Initialize all variational parameters.
\FOR{iteration $t=1,\dots,T_{\max}$}
    \FOR{each user $i$}
        \STATE Update $m_{\theta_i}$ and $V_{\theta_i}$.
    \ENDFOR
    \FOR{each item $j$}
        \STATE Update $m_{\beta_j}$ and $V_{\beta_j}$.
    \ENDFOR
    \FOR{each user $i$}
        \STATE Update $m_{b_i}$.
    \ENDFOR
    \FOR{each item $j$}
        \STATE Update $m_{b_j}$.
    \ENDFOR
    \STATE Optionally evaluate validation performance.
\ENDFOR
\end{algorithmic}
\end{algorithm}

% ============================================================
% 3. POISSON MATRIX FACTORIZATION (PMF)
% ============================================================

\section{Poisson Matrix Factorization}

\subsection{Model Definition}

We now consider a non-Gaussian model suitable for sparse, non-negative observations such as implicit feedback or count data.  
Given $U$ users, $M$ items, and latent dimension $K$, each observed entry $x_{ij} \ge 0$ is modeled as
\[
x_{ij} \mid \theta_i, \beta_j \sim \text{Poisson}(\theta_i^\top \beta_j).
\]

We place independent Gamma priors over all latent factors:
\[
\theta_{ik} \sim \text{Gamma}(a_0, b_0), \qquad
\beta_{jk} \sim \text{Gamma}(a_0, b_0),
\]
where the Gamma distribution is parameterized by shape $a$ and rate $b$, so that  
$\mathbb{E}[\theta_{ik}] = a_{\theta,ik} / b_{\theta,ik}$.

As before, we use the index sets
\[
\Omega_i = \{ j : (i,j)\ \text{observed} \},
\qquad
\Omega^j = \{ i : (i,j)\ \text{observed} \}.
\]

\subsection{Variational Family}

Under the mean-field assumption, the variational distribution factorizes as
\[
q =
\prod_{i=1}^U \prod_{k=1}^K q(\theta_{ik})
\prod_{j=1}^M \prod_{k=1}^K q(\beta_{jk}),
\]
with each factor being a Gamma distribution:
\[
q(\theta_{ik}) = \text{Gamma}(a_{\theta,ik}, b_{\theta,ik}),
\qquad
q(\beta_{jk}) = \text{Gamma}(a_{\beta,jk}, b_{\beta,jk}).
\]

Expectations needed for CAVI are simply
\[
\mathbb{E}_q[\theta_{ik}] = \frac{a_{\theta,ik}}{b_{\theta,ik}}, \qquad
\mathbb{E}_q[\beta_{jk}] = \frac{a_{\beta,jk}}{b_{\beta,jk}}.
\]

\subsection{Complete Conditionals}

Unlike the Gaussian MF case, the Poisson likelihood is not conjugate to a gamma distribution unless we introduce latent allocation variables. However, the standard collapsed CAVI updates for Poisson MF exploit the identity
\[
x_{ij} = \sum_{k=1}^K x_{ijk},
\qquad
x_{ijk} \mid \theta_i, \beta_j \sim \text{Poisson}(\theta_{ik}\beta_{jk}),
\]
together with the conditional expectation
\[
\mathbb{E}[x_{ijk} \mid x_{ij},\theta_i,\beta_j]
=
x_{ij}\;
\frac{\theta_{ik}\beta_{jk}}
{\sum_{\ell=1}^K \theta_{i\ell}\beta_{j\ell}}.
\]

Taking expectations under the current variational distribution yields
\[
\mathbb{E}_q[x_{ijk}]
=
x_{ij}
\frac{\mathbb{E}_q[\theta_{ik}]\,\mathbb{E}_q[\beta_{jk}]}
{\sum_{\ell=1}^K 
\mathbb{E}_q[\theta_{i\ell}]\,\mathbb{E}_q[\beta_{j\ell}] }.
\]

Plugging these into the Gamma–Poisson conjugacy, the complete conditionals are
\[
a_{\theta,ik}^{\text{new}}
=
a_0 + \sum_{j\in\Omega_i} \mathbb{E}_q[x_{ijk}],
\qquad
b_{\theta,ik}^{\text{new}}
=
b_0 + \sum_{j\in\Omega_i} \mathbb{E}_q[\beta_{jk}],
\]
\[
a_{\beta,jk}^{\text{new}}
=
a_0 + \sum_{i\in\Omega^j} \mathbb{E}_q[x_{ijk}],
\qquad
b_{\beta,jk}^{\text{new}}
=
b_0 + \sum_{i\in\Omega^j} \mathbb{E}_q[\theta_{ik}].
\]

\subsection{CAVI Updates}

Let
\[
E_{\theta,ik} = \mathbb{E}_q[\theta_{ik}],\qquad
E_{\beta,jk} = \mathbb{E}_q[\beta_{jk}].
\]

For each observed $(i,j)$, define the variational rate estimate
\[
\lambda_{ij}
=
\sum_{k=1}^K E_{\theta,ik} E_{\beta,jk}.
\]

Then the expected factor allocation is
\[
\zeta_{ijk}
=
x_{ij}
\frac{E_{\theta,ik} E_{\beta,jk}}
{\lambda_{ij}},
\qquad
\text{for } k = 1,\dots,K.
\]

\subsubsection{Updates for user factors}

\[
a_{\theta,ik}^{\text{new}}
=
a_0 + \sum_{j\in\Omega_i} \zeta_{ijk},
\qquad
b_{\theta,ik}^{\text{new}}
=
b_0 + \sum_{j\in\Omega_i} E_{\beta,jk}.
\]

\subsubsection{Updates for item factors}

\[
a_{\beta,jk}^{\text{new}}
=
a_0 + \sum_{i\in\Omega^j} \zeta_{ijk},
\qquad
b_{\beta,jk}^{\text{new}}
=
b_0 + \sum_{i\in\Omega^j} E_{\theta,ik}.
\]

After each update, expectations are recomputed as
\[
E_{\theta,ik} = \frac{a_{\theta,ik}}{b_{\theta,ik}},
\qquad
E_{\beta,jk} = \frac{a_{\beta,jk}}{b_{\beta,jk}}.
\]

\subsection{Pseudocode}

\begin{algorithm}[H]
\caption{CAVI for Poisson Matrix Factorization}
\begin{algorithmic}[1]
\STATE Initialize Gamma variational parameters $(a_{\theta},b_{\theta})$, $(a_{\beta},b_{\beta})$.
\FOR{iteration $t=1,\dots,T_{\max}$}
    \STATE Compute expectations $E_{\theta}$ and $E_{\beta}$.
    \FOR{each user $i$}
        \FOR{each observed item $j\in\Omega_i$}
            \STATE Compute $\lambda_{ij} = \sum_k E_{\theta,ik} E_{\beta,jk}$.
            \STATE Compute allocation $\zeta_{ijk} = x_{ij}(E_{\theta,ik}E_{\beta,jk})/\lambda_{ij}$.
        \ENDFOR
        \STATE Update $a_{\theta,i}$ and $b_{\theta,i}$.
    \ENDFOR
    \FOR{each item $j$}
        \STATE Update $a_{\beta,j}$ and $b_{\beta,j}$ analogously.
    \ENDFOR
    \STATE Optionally evaluate validation RMSE.
\ENDFOR
\end{algorithmic}
\end{algorithm}


\end{document}